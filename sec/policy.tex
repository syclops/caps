\section{Policy}
\label{sec:policy}

In this section, we provide a detailed description of policy and log operations
in \ac{name}. We begin with a brief overview of how logs in \ac{ct} work and
their shortcomings as they apply to \ac{name}. We then describe each of the main
operations of logs in \ac{name}: initial registration, updates, overwriting, and
proofs.

\subsection{CT Log Background}

Public logs in \ac{ct} are responsible for publicizing the certificates that
\acp{ca} issue. According to the \ac{ct} specification, logs offer the following
functionality:

\begin{compactitem}
\item \texttt{submit-entry}: submit a certificate or precertificate (described
  below) along with a certificate chain to the log for inclusion.
\item \texttt{get-sth}: retrieve a signed and timestamped copy of the
  latest root hash of the log's Merkle hash tree, called \iac{sth}.
\item \texttt{get-sth-consistency}: retrieve a consistency proof given two
  \acp{sth} of the log.
\item \texttt{get-proof-by-hash}: retrieve an inclusion proof for a given leaf
  hash and desired Merkle hash tree size (representing the state of a log at a
  given point).
\item \texttt{get-all-by-hash}: retrieve an inclusion proof, \ac{sth}, and
  consistency proof given a leaf hash and desired tree size.
\item \texttt{get-entries}: given a start and end index, retrieve a range of
  entries (including the certificate and their \acp{sct}) as well as the latest
  \ac{sth}.
\item \texttt{get-anchors}: retrieve a list of the trust anchors of the log
  (that is, the set of certificates that all submitted certificate chains must
  be rooted in).
\end{compactitem}

\ac{tls} clients only need \acp{sct} to deem a certificate valid in \ac{ct}, and
the number of \acp{sct} required depends on the lifetime duration of the
certificate \steve{cite CT log plan}. Therefore, for those clients not
performing monitoring or auditing functions, not even the \ac{sth} is
necessary. This is because \ac{ct} only cares about the fact that a potentially
unauthorized certificate was publicized \emph{at some point}, and as long as the
log is operating correctly (i.e., including the certificate in its Merkls hash
tree after issuing \iac{sct}), misbehavior \emph{should} be detected at some
point.

\subsection{Logs in \ac{name}}

Logs in \ac{name} maintain a mapping between domains and their policy
information, in addition to the information they maintain as \ac{ct} logs.

\subsection{Initial Policy Registration}

The initial policy registration can be performed when the log has no entry for
the given domain. A log performs the initial policy registration when its
receives a certificate whose subject name does not match any domain that the log
has in its mapping. In this case, the log simply creates a new policy entry for
that domain and associates the submitted certificate's public key with the
domain. The policy value is set to be the default value of 1 (indicating that
a client should expect only a single logged certificate chain from the domain).

Each new public key or certificate needs to be bound to a policy ID as well as
to the domain itself. It should be impossible to bind a new key to a policy ID
without the approval of the policy holder. Thus the domain will need to provide
some additional information regarding its authorization of the addition or
removal of any public key or certificate.

We therefore want the policy ID to be stable or at least linkable given an
addition or deletion operation. We also want the ID to be used to authenticate
a signature authorizing the addition or deletion of a key.

\subsection{Policy Updates}

\steve{Policies can be strengthened (i.e., not updated in terms of content but
in terms of independent chains) by simply specifying the same policy.}

\subsection{Policy Overwriting}

\subsection{Log Policy Proofs}

\steve{The irrelevance of log inclusion or consistency proofs to clients does
not hold in \ac{name}. This is because clients need to be convinced that a
policy was logged and that no policy overwrites or updates have taken place
since (i.e., the client sees the latest version of the policy). (Actually, if
the policy was logged and is not the latest, that could still prevent many
\acp{ca} from issuing unauthorized certificates if using the authorized \ac{ca}
approach. If we use the number of independent chains approach, we could include
a Bloom filter--like structure that makes it difficult for an adversary to get
away with fewer certificates even by forging them. How secure are Bloom filters
to collision attacks, anyway?)}

\subsection{Other Operations}

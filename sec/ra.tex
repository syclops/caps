\section{Registration Authority}
\label{sec:ra}

\subsection{Policy}
\label{sec:ra:policy}

In a nutshell, a policy associates a domain with the public keys it uses during
the TLS handshake; a policy thus effectively pins a set of public keys to a
domain. Specifically, a policy binds a domain name to a set of \emph{policy
public keys}, an \emph{update threshold}, a \emph{registration proof}, and a set
of \emph{authentication public keys}. We describe each of these components
below.

The policy public keys and update threshold are used to authenticate updates to
the domain's policy. Specifically, to update its policy, a domain must provide a
threshold number of signatures on the new desired policy, and each of these
signatures must successfully verify with a distinct policy public key. Such an
update mechanism allows each domain to tune its policy's security to a desired
level: a lower update threshold means that an adversary needs to compromise
fewer policy private keys to maliciously update a domain's policy, while a
higher update threshold means that a domain must generate more signatures
(resulting in more computation) and send more information (resulting in higher
bandwidth cost) to update its policy.

The registration proof is used when the domain registers or recovers its policy.
A proof consists of one or more certificate chains that meet certain criteria
and a signature on the new desired policy. The RA maintains a list of root keys
in which each of these chains must be anchored. A proof also has an attribute
called \emph{strength}, which represents the number of keys that must be
compromised to forge all certificate chains. As we describe below, the strength
of a proof can be efficiently computed.

Finally, the authentication public keys are a set of whitelisted keys for
clients. In other words, a client will not accept a public key as belonging to a
domain if it does not appear in the domain's set of authentication public keys.
Whitelisting keys rather than certificates means that the domain does not have
to change its policy each time it obtains a new certificate. \steve{how often do
keys change between certificates from the same CA?}

%The use of domain-specified policies to provide authentication information is
%not new. DANE~\cite{rfc6698} allows domains to communicate similar policies via
%DNS records. However, DANE relies on DNSSEC (otherwise, the records containing
%these policies cannot be authenticated), and does not enable trust agility:
%clients must trust DNS to validate a DANE record.
%PoliCert~\cite{szalachowski2014policert} provides a slightly richer range of
%policies, but also lacks 

\subsection{Registration Proof Strength}
\label{sec:ra:proof}

The strength of a registration proof can be efficiently calculated. For the
purposes of computing strength, we represent a registration proof as a set of
chains $\{C_1, \ldots, C_n\}$. Each chain can be represented as a sequence of
keys $C_i = (K_i^1, \ldots, K_i^{\ell_i})$ where $K_i^1 \in R$ (assuming that
$R$ is the set of the root keys allowed by the RA) and $K_i^{\ell_i} = K_D$. We
then transform this set of chains into a graph as follows: the set of vertices
$V$ consists of a source node $S$ along with two nodes for each $K$ where $K \in
C_i$ for some $i$ such that $1 \le i \le n$. For a given $K$, we label these
nodes as $K$ and $K'$. The set of labeled edges $E$ is defined as follows: $(S,
K_i^1) \in E$ for all $i$ where $1 \le i \le n$, and all such edges are labeled
with $\infty$. In addition, for all $i$ such that $1 \le i \le n$ and all $j$
such that $1 \le j < \ell_i$, $(K_i^j, {K_i^j}') \in E$ and all such edges are
labeled with 1. Finally, for all $i$ such that $1 \le i \le n$ and all $j$ such
that $1 \le j < \ell_i$, $({K_i^j}', K_i^{j+1}) \in E$ and all such edges are
labeled with $\infty$.

We can then consider these edge labels as capacities in a minimum cut problem
with $S$ as the source and $K_D$ as the sink. The minimum cut will go through
some number of edges labeled with 1, and each of these edges represents a key
that needs to be compromised in order to disconnect all certificate chains. Thus
the minimum cut in this graph represents the minimum number of keys that must be
compromised in order to forge the entire set of certificate chains.

\subsection{RA Overview}
\label{sec:ra:overview}

The RA maintains a mapping of domain names to policies, as well as information
pertaining to domain policy registration. Specifically, the RA has a list of
accepted root public keys for registration proofs, as well as a mapping of
domain names to their most recent registration proofs.

The RA offers an API consisting of five methods, covering initial registration,
updates using policy keys, recovery using a stronger registration proof, lookup,
and enumeration of the RA database. We describe each below. \steve{Add
algorithmic descriptions}

The \texttt{register} method takes as input a domain name, a registration proof
consisting of a set of certificate chains and a signature on the desired policy,
and the desired policy itself. The domain name cannot already be present in the
RA's mapping. For each certificate chain, the RA checks that the leaf key can be
used to verify the signature on the desired policy, that each certificate has a
valid signature and has not expired, and that the chain is rooted in the RA's
set of root public keys. The RA then computes the strength of the proof and
associates the desired policy with the domain name. It also stores the
registration proof and strength, associated with the domain name.

The \texttt{update} method takes as input a domain name, the new desired policy,
and a set of signatures on the above information. The RA checks that each
signature can be verified with a distinct policy public key, and that the number
of signatures is at least the update threshold. If these checks pass, the RA
stores the new policy in its domain-policy mapping.

The \texttt{recover} method takes as input a domain name, a new desired policy,
and a registration proof. The domain name must be present in the RA's mapping.
The RA checks the registration proof in the same way as for initial
registration, but also checks that the proof is stronger than the current
strength of the domain's proof. If this check passes, the RA associates the new
policy with the domain name and updates the domain's registration proof
strength. We note that the new desired policy can be the same as the domain's
current policy, thus allowing a domain to ``strengthen'' its own registered
policy.

The \texttt{lookup} method takes as input a domain name and returns its policy,
registration proof, and proof strength. The domain name must be present in the
RA's mapping.

The \texttt{enumerate} methods takes no inputs and returns a list of all domain
names registered at the RA.

\documentclass{article}

\usepackage{amsthm}
\usepackage{amssymb}
\usepackage{xspace}

\newcommand{\emxs}[1]{\ensuremath{#1}\xspace}

\newcommand{\alphabet}{\emxs{\Sigma}}
\newcommand{\edges}{\emxs{E}}
\newcommand{\immutlabels}{\emxs{L_I}}
\newcommand{\labelfunc}{\emxs{L}}
\newcommand{\nats}{\emxs{\mathbb{N}}}
\newcommand{\vertices}{\emxs{V}}
\newcommand{\strings}{\emxs{\alphabet^*}}

\theoremstyle{definition}
\newtheorem{definition}{Definition}

\title{A Formulation of the\\DAWG Node Compaction Decision Problem}
\author{Steve Matsumoto}

\begin{document}

\maketitle

In this document, I will provide a formulation of the problem of compacting
nodes (and their labels) in a directed acyclic word graph (DAWG), a finite state
automaton that can recognize a lexicon of words (domain names in our
application).

We begin by defining a DAWG:

\begin{definition}
  Let \alphabet be an alphabet of symbols. A \emph{directed acyclic word graph}
  is a triple $(\vertices, \edges, \labelfunc)$ where \vertices is a set, \edges
  is a multiset of elements in $\vertices^2$ (i.e., ordered pairs of vertices),
  and $\labelfunc: \edges \to \strings$ is a function. We call \vertices the set
  of \emph{vertices} of the DAWG, \edges the set of \emph{edges}, and \labelfunc
  is the \emph{label function}.
  \label{def:dawg}
\end{definition}

As a side note, we assume that the edges in \edges are distinguishable, that is,
that multiple occurrences of a pair of vertices are treated as though they were
unique elements. (This can easily be done by identifying edges as elements of
the set $\vertices \times \vertices \times \nats$, where the natural number
component is used to identify a specific edge between a pair of vertices.)

For the purposes of formulating the problem, we first provide some useful
definitions based on that of our DAWG:

\begin{definition}
  Let $(\vertices, \edges, \labelfunc)$ be a DAWG, and let $v$ be a vertex in
  \vertices. The set of \emph{in-edges} of $v$ is defined as the set
  \[
    \edges_i(v) = \left\{ (x, v) \in \edges: x \in \vertices \right\},
  \]
  and the set of \emph{out-edges} of $v$ is defined as the set
  \[
    \edges_o(v) = \left\{ (v, x) \in \edges: x \in \vertices \right\}.
  \]
  The \emph{in-degree} of $v$ is defined as the quantity $d_i(v) =
  |\edges_i(v)|$, and the \emph{out-degree} of $v$ is defined as $d_o(v) =
  |\edges_o(v)|$.
\end{definition}

\begin{definition}
  Let $(\vertices, \edges, \labelfunc)$ be a DAWG, and let $v$ be a vertex in
  \vertices. For $n \ge 2$, we say that $v$ is \emph{$n$-compactable} if either
  $d_i(v) = 1$ and $1 < d_o(v) \le n$, or $1 < d_i(v) \le n$ and $d_o(v) = 1$.

  We define $\vertices_n$ as the set of all vertices in \vertices that are
  $n$-compactable.
\end{definition}

\begin{definition}
  Let $(\vertices, \edges, \labelfunc)$ be a DAWG, and let $S$ be a set of
  vertices such that $S \subseteq \vertices$. The set of \emph{incident edges to
  $S$} is defined as the set
  \[
    \edges_S = \left\{ (x, y) \in \edges: x \in S \lor y \in S \right\}.
  \]
\end{definition}

\begin{definition}
  Let $(\vertices, \edges, \labelfunc)$ be a DAWG, and let $F$ be a multiset of
  edges such that $F \subseteq \edges$. The set of \emph{labels of $F$} is
  defined as the set
  \[
    \labelfunc(F) = \left\{ s \in \strings: \exists f \in F\ |\ \labelfunc(f) = s
    \right\}.
  \]
\end{definition}

When we carry out node compaction, edges incident to the compacted node will be
eliminated, and each incoming edge's label will be concatenated with each
outgoing edge's label. Therefore, we can consider for each possible node to
compact the labels that may be eliminated or introduced as a result of the
compaction. However, when we consider this, we note that for all edges not
incident on an $n$-compactable node, their labels cannot be eliminated (because
that edge cannot be eliminated), and their labels cannot be introduced (since
they are already in the DAWG by definition). Therefore, we can define the
following:

\begin{definition}
  Let $(\vertices, \edges, \labelfunc)$ be a DAWG, and let
  $\edges_{\vertices_n}$ be the set of all edges incident on $n$-compactable
  vertices in the DAWG (as specified by previous definitions). We define the set
  of \emph{immutable labels} as the set $\immutlabels = \labelfunc(\edges
  \setminus \edges_{\vertices_n})$.
\end{definition}

With this definition in place, we see that the labels that could possibly be
completely eliminated from the DAWG by compacting a subset of $n$-compactable
nodes is simply $\labelfunc(\edges_{\vertices_n}) \setminus \immutlabels$.
To consider the labels that might be introduced, we need to construct
concatenations of labels as mentioned above. Therefore, we define the following:

\begin{definition}
  Let $(\vertices, \edges, \labelfunc)$ be a DAWG, and let $S \subseteq
  \vertices$. The set of
  \emph{compaction-generated labels} is given as the set
  \[
    P_S = \left\{ s \in \strings: \exists v \in S, e_i \in \edges_i(v), e_o \in
    \edges_o(v)\ |\  \labelfunc(e_i) \| \labelfunc(e_o) = s \right\}
  \]
  where $\|$ denotes string concatenation.
\end{definition}

Again, the set of labels that might be introduced is actually $P_{\vertices_n}
\setminus \immutlabels$. Also note that there may still be overlap between
$\labelfunc(\edges_{\vertices_n}) \setminus \immutlabels$ and $P_{\vertices_n}
\setminus \immutlabels$.

To formulate our problem, we construct a bipartite graph where edges are between
the sets $\vertices_n$ and $\labelfunc_n = (\labelfunc(\edges_{\vertices_n})
\setminus \immutlabels) \cup (P_{\vertices_n} \setminus \immutlabels)$. We
partition the edges into two sets, $\edges_a$ and $\edges_r$, representing the
set of labels that may possibly be introduced or removed from the DAWG. Thus
$(x, y) \in \edges_a$ if there exist edges $e_i \in \edges_i(x)$ and $e_o \in
\edges_o(x)$ such that $\labelfunc(e_i) \| \labelfunc(e_o) = y$, and $(x, y) \in
\edges_r$ if $y \in \edges_i(x) \cup \edges_o(x)$.

Before we provide the formal problem statement, we define the following:

\begin{definition}
  Let $E$ be a set of edges in the bipartite graph above, and let $S$ be a
  subset of $\vertices_n$ or of $\labelfunc_n$. We define the set $N_E(S)$ to be
  the set of \emph{neighbors} of vertices in $S$ in the graph.
\end{definition}

The problem is then to select a subset $S \subseteq \vertices_n$ in order to
minimize the following quantity:
\[
  -2|S| + |N_{\edges_a}(S) \setminus \labelfunc(\edges)| - \left|\left\{
    \ell \in N_{\edges_r}(S) \setminus N_{\edges_a}(S): N_{\edges_r}(\{\ell\})
    \subseteq S \right\}\right|
\]
The first term in this quantity is derived from the fact that compacting a
vertex saves one vertex and one edge. The second term captures the number of new
labels introduced to the DAWG as a result of compaction (hence the set
difference from labels that are removed, since that means they are already in
the DAWG). Note here that the set difference takes into account labels removed
by \emph{any} vertex, even those not in $S$, to capture the fact that any label
removed is already in the DAWG. Finally, the third term captures the number of
labels eliminated from the DAWG, again taking the set difference with labels
introduced by compacting nodes in $S$, since such an introduction would mean
that not all instances of the label can be eliminated. The conditional criterion
in the third term simply captures the fact that compacting the nodes in set $S$
would eliminated all instances of the label.

\end{document}

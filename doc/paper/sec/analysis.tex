\section{Security Analysis}
\label{sec:analysis}

In this section, we analyze the security that \ac{name} provides. We begin with
our main security claim, that an adversary must compromise a threshold number
of \acp{ca} or log aggregators to mount a successful \ac{mitm} attack, and
informally argue that it holds given our adversary model. We then describe
potential weaknesses of \ac{name} not covered by our security claim, and argue
why these weaknesses are unlikely to be used in practice, able to be mitigated
by simple countermeasures, or out of scope for \ac{name}.

\subsection{Main Security Claim}
\label{sec:analysis:informal}

In making our main security claim, we assume that the client requests \numlas
policy proofs and that the domain has a domain name of \domain and a policy
value of \policy. 
%We assume that both the client and domain follow the handshake protocol as described. 
We also assume that our adversary controls $\numlas - 1$
private keys of log aggregators and the private keys of $\policy - 1$ \acp{ca}.
The adversary can also, of course, create its own public key pairs, e.g., 
$(\pk_\adversary, \pk_\adversary^{-1})$.
%and we assume that the adversary is the sole entity that controls the private
%key corresponding to a public key $\pk_\adversary$. 
We assume that the adversary
can intercept, suppress, replay, and modify any handshake message sent between
the client and domain, and can use its private keys to 
sign any message it can construct with the
information it obtains. We also assume that the adversary
cannot obtain a certificate binding \domain to $\pk_\adversary$ from any \ac{ca}
besides the ones it controls. Under these assumptions, we claim that a \ac{name}-enabled client
will abort the handshake protocol if it receives any certificate chain
containing $\pk_\adversary$ as the leaf public key. This claim 
supports the security of \ac{name} because to mount a successful MITM attack,
the adversary must convince the client to complete the \ac{name} handshake
based on a key that the adversary controls.

We first show that the adversary cannot convince the client that that the domain
has a policy value other than \policy. Because the client requests \numlas
policy proofs for \domain, the client will abort the handshake if the 
ServerHello message does not contain \numlas independent policy
proofs. Because we assume that the adversary only has access to $\numlas - 1$
log aggregator signing keys, the adversary cannot use those keys to
generate \numlas independent proofs. Specifically, if the adversary sends a set
of proofs and there are fewer than \numlas valid proofs, or if any of them fails
to prove that \policy is the policy value for \domain, then the client will
abort the handshake.

%Recall that the client determines \policy as the one plus the number of valid,
%independent certificate chains sent in the extension message. We show that the
%adversary cannot convince the client that the $\pk_\adversary$ is the domain's
%public key. From our assumption that the adversary can access the signing keys
%of $\policy - 1$ \acp{ca}, we know that the adversary controls \emph{at least}
%$\policy - 1$ private keys. 

From our assumption that the adversary can access the signing keys
of $\policy - 1$ \acp{ca},
we know that the number of \emph{independent} certificate chains that the adversary can generate
is \emph{at most} $\policy - 1$. It is straightforward to show this by induction
on \policy, with $\policy = 2$ as the base case.\footnote{Note that the $\policy
= 1$ case is equivalent to the current Web \ac{pki} (i.e., the domain only
provides a single certificate chain).} If the adversary generates $\policy - 1$
independent certificate chains for $\pk_\adversary$ and sends these chains to
the client, the client will validate the chains, but abort the handshake when
the $\policy^{th}$ independent chain fails to arrive. 

\subsection{Potential Weaknesses}
\label{sec:analysis:weaknesses}

Our main security claim shows that under our assumptions, the adversary cannot
mount a successful \ac{mitm} attack on a client and domain. However, we did not
yet address the ways in which \acp{ca}, log aggregators, or other parties shown
in \autoref{fig:overview} may fail. We now discuss possible failures for each
of these parties and how they may affect the security of \ac{name}, along with
potential mitigations.

\paragraph{\acp{ca}}

Historically, all publicly known CA failures have been singletons (i.e.,
separated in time and causation from other failures), but a systemic flaw may
allow an adversary to compromise many \acp{ca} at once. 
Such a widespread compromise would be quickly detected by public
logs and certificate scanning services. Subsequently, the browsers or \acp{ca}
could issue revocations of the affected certificates or \ac{ca} keys using
existing methods.

\paragraph{Public Logs}

A misbehaving public log may record a fraudulently-issued certificate, refuse
to include a certificate in its database, or attempt to change details of
previously logged certificates. \ac{ct} expects auditors (a role anyone can adopt) to
monitor logs for any such misbehavior~\cite{rfc6962}. 
Recording fraudulent certificates may cause a log
aggregator to compute a policy value that is too high, making a domain
inaccessible in \ac{name}. However, this also requires the failure of $\policy$
\acp{ca} to mount \iac{mitm} attack, and would be quickly detected by an
auditor. Once detected, browsers and log aggregators can simply ignore the
misbehaving log.

\paragraph{Log Aggregators}

A log aggregator's private key may become lost or stolen, allowing it to
provide incorrect policy information or signaling sets. A successful \ac{mitm}
attack would require $\numlas - 1$ aggregators to send incorrect policy
information. Clients can use a sufficiently high value of $\numlas$ to minimize
this risk, and can also take a similar approach to verify signaling sets and
their updates.

A log aggregator may not provide updates in a timely manner, leading to policy
values that are incorrect or mismatched with those of other aggregators, which
in turn can make a domain inaccessible or provide insufficient protection
against misbehaving \acp{ca}. Since domains provide policy proofs to clients,
we expect that domains can quickly detect this delay and send the correct
number of certificates or fix the issue with the relevant aggregators. The
delay may also affect signaling sets, but only if the policy value changes
between $\nohttps$, $\onecert$, and $\multicert$ (0, 1, and $\ge 2$ independent
chains, respectively). Finally, as logs themselves are held to a standard of
availability, doing the same for log aggregators can minimize the risk of these
failures.

\paragraph{Browser Vendors}

A misbehaving browser vendor can ship a malicious version of a browser to
clients, and we consider this failure mode outside of the scope of \ac{name}. A
malicious browser can arbitrarily deviate from \ac{tls} or \ac{name}, insert
its own root certificates, or display arbitrary pages. 
%Even with \ac{name}, an
%adversary-controlled browser can also modify queries or responses of the
%signaling set or carry out \ac{tls} stripping attacks. While some
%countermeasures such as OS-level protections may protect the client, we suggest
%that defenses against PKI failures (whether \ac{name} or others) are unlikely
%to stop these powerful client-side adversaries.

%\paragraph{Resource Exhaustion}

%Log aggregators rely on public logs to obtain a complete view of certificates in
%the Web \ac{pki}. However, given the open nature of many of these logs (i.e.,
%the fact that many parties can submit or otherwise make
%certificates available to them), an adversary may take advantage of this
%openness to force the log aggregators to increase their resource consumption.
%For example, an adversary who controls \iac{ca} signing key can issue millions
%of certificates and make them available to the logs, causing the log aggregators
%to store these certificates when updating their respective policy databases.

%While \ac{name} provides no mechanism to stop this behavior, the issuance of
%these certificates produces traceable evidence of the adversary's behavior,
%and the requirement that newly-issued certificates be logged increases the
%likelihood of someone detecting this behavior. This detection would likely lead
%to consequences such as the revocation of the \ac{ca} certificate by its issuing
%\ac{ca} or by the browser vendor. Moreover, only an adversary with access to
%\iac{ca} signing key can carry out this attack; while Let's Encrypt does provide
%free certificates through an automatic protocol, it imposes a rate limit per
%domain based on the Public Suffix
%List,\footnote{\url{https://publicsuffix.org/}} thus requiring an adversary
%without access to \iac{ca} signing key to obtain many domain names to carry out
%this attack.


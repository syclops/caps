\section{Policy}
\label{sec:policy}

In this section, we provide a detailed description of policy and log operations
in \ac{name}. We begin with a brief overview of how logs in \ac{ct} work and
their shortcomings as they apply to \ac{name}. We then describe each of the main
operations of logs in \ac{name}: initial registration, updates, overwriting, and
proofs.

\subsection{Policy Design}

Simplicity and minimal domain involvement lie at the heart of policies in
\ac{name}. We observe that systems such as ARPKI~\cite{basin2014arpki} and
PoliCert~\cite{szalachowski2014policert} \emph{require} domain involvement for
each policy operation, including initially specifying a certificate policy and
updating the policy in the future. While these systems (reasonably) assume that
domains only infrequently change their policies, the need to involve each of
potentially millions of domains even for its initial steps hinders
deployability. We thus argue that a deployable solution to the certificate
policy problem should minimize the number of domains that interact directly with
the policy infrastructure. At the same time, a domain who has a policy should
stay protected from one-off \ac{ca} failures in the sense that a single
misissued certificate is no longer sufficient to enable a \ac{mitm} attack
against the domain. Our policy design thus aims to transition the existing
\ac{pki} from weakest-link security towards ``weakest-$n$-links'' security.

The need to minimize domain involvement precludes a generalized policy system.
Even relatively simple policies, such as those specifying a list of trusted
\acp{ca}, requires domain involvement for each change in policy because they
rely on information that only domains can authoritatively provide. In today's
Web \ac{pki}, domains can communicate information about their public keys
through only two actions: purchasing certificate from \acp{ca}, and requesting
that these certificates be revoked. We observe, however, that domains can use
these two actions (with which all \ac{https}-enabled domains are well familiar)
to implicitly provide information about their policies. In particular, a domain
can communicate which public key clients should accept by providing multiple
certificates for that public key. Then, to assess the trustworthiness of a
public key associated with a domain, we can consider the number of \acp{ca} that
would have had to collude to certify that key, and have clients accept only the
keys for which the most \acp{ca} would have had to misbehave.

This criterion, though simple, is powerful and easy to deploy. A domain can
defend against \ac{mitm} attacks by any single misbehaving \ac{ca} by simply
obtaining two independent certificates for its public key. Each domain can tune
its level of resilience against colluding compromised \acp{ca} by simply
obtaining more certificates for its key. Moreover, a domain does not need to do
anything other than obtain multiple certificates for its desired key to increase
its key's strength. In the event that a domain wants to have multiple public
keys, it can simply obtain the same number of certificates for each of its
desired keys. Barring private key compromise, \iac{mitm} attack can only take
place if the adversary obtains more certificates for an alternate public key,
which would require concerted effort from multiple \acp{ca}. Therefore, our
policy design that assigns greater trustworthiness to keys with more
certificates issued for them provides strong, deployable security against
misbehaving \acp{ca}.

Formally, our policy design can be stated as follows. Let each member of the set
$\{\certchain_1, \ldots, \certchain_n\}$ represent a certificate chain, written
as $\certchain_i = (\pk_i^1, \ldots, \pk_i^{\ell_i})$, where each key in the
chain certifies the next. Supposing that a public log has the set \rootset as
its trusted root \ac{ca} keys, we say that a chain is \emph{rooted} in \rootset
if $\pk_i^1 \in \rootset$, and we say that a set of certificate chains
$\{\certchain_1, \ldots, \certchain_n\}$ \emph{collectively certifies} a public
key \pk if for all $i$ such that $1 \le i \le n$, $\pk_i^{\ell_i} = \pk$. We
then define the \emph{strength} of a set of certificate chains as the number of
private keys corresponding to a public key in the set of chains that must be
compromised to collectively certify a different public key $\pk'$. We
observe that the strength of a set of $n$ certificate chains is at most $n$.

We can compute the strength of a set of certificate chains by transforming the
set into a graph as follows: the set of vertices consists of a source node $S$
along with two nodes for each public key $\pk$ where $\pk \in \certchain_i$ for
some $i$ such that $1 \le i \le n$. For a given $\pk$, we label these nodes as
$\pk$ and $\pk'$. The set of weighted edges $E$ is defined as follows: $(S,
\pk_i^1) \in E$ for all $i$ where $1 \le i \le n$, and all such edges have a
weight of $\infty$. In addition, for all $i$ such that $1 \le i \le n$ and all
$j$ such that $1 \le j < \ell_i$, $(\pk_i^j, {\pk_i^j}') \in E$ and all such
edges have a weight of 1. Finally, for all $i$ such that $1 \le i \le n$ and all
$j$ such that $1 \le j < \ell_i$, $({\pk_i^j}', \pk_i^{j+1}) \in E$ and all such
edges have a weight of $\infty$. Then, the strength of the set of certificate
chains is equal to the min-cut of this graph.

Each time a certificate (along with its certificate chain) for a given domain is
logged at a \ac{name} log, the log examines the strength of the sets of
certificate chains for each key observed for that domain. The log then assigns
the policy for the domain as the highest strength of any certificate chain for
that domain. \ac{name}-compatible clients then expect a minimum of that many
certificate chains, along with a log proof for that level of strength, from the
domain during the \ac{tls} handshake.

\subsection{\ac{name} Logs}

\begin{table}[t]
  \centering
  \caption{\ac{name} log API. \steve{TODO: add enumeration and policy retrieval
  functions.}}
  \begin{tabularx}{\linewidth}{|l|X|}
  \toprule
  \textbf{Function} & \textbf{Use}\\
  \midrule
  \texttt{submit-entry}
  & submit a certificate or precertificate (described below) along with a
  certificate chain to the log for inclusion.\\
  \midrule
  \texttt{get-sth}
  & retrieve a signed and timestamped copy of the latest root hash of the log's
  Merkle hash tree, called \iac{sth}.\\
  \midrule
  \texttt{get-sth-consistency}
  & retrieve a consistency proof given two \acp{sth} of the log.\\
  \midrule
  \texttt{get-proof-by-hash}
  & retrieve an inclusion proof for a given leaf hash and desired Merkle hash
  tree size (representing the state of a log at a given point).\\
  \midrule
  \texttt{get-all-by-hash}
  & retrieve an inclusion proof, \ac{sth}, and consistency proof given a leaf
  hash and desired tree size.\\
  \midrule
  \texttt{get-entries}
  & given a start and end index, retrieve a range of entries (including the
  certificate and their \acp{sct}) as well as the latest \ac{sth}.\\
  \midrule
  \texttt{get-anchors}
  & retrieve a list of the trust anchors of the log (that is, the set of
  certificates that all submitted certificate chains must be rooted in).\\
  \bottomrule
\end{tabularx}

  \label{tab:api}
\end{table}

Public logs in \ac{name} extend those of \ac{ct}. They thus offer the API shown
in \autoref{tab:api}.

\ac{name} logs store certificates they have observed as in \ac{ct}. In addition,
logs in \ac{name} also store for each domain name a map of all public keys
associated with that domain, along with the maximum strengths of all sets of
certificate chains for each public key. Logs use this map to create a
\emph{policy tree}, a Merkle hash tree whose leaves are lexicographically
ordered in the reverse domain name (i.e., the domain name with the labels in
reverse order) and store the policy (i.e., the number of signatures a client
should expect).

%\ac{tls} clients only need \acp{sct} to deem a certificate valid in \ac{ct}, and
%the number of \acp{sct} required depends on the lifetime duration of the
%certificate \steve{cite CT log plan}. Therefore, for those clients not
%performing monitoring or auditing functions, not even the \ac{sth} is
%necessary. This is because \ac{ct} only cares about the fact that a potentially
%unauthorized certificate was publicized \emph{at some point}, and as long as the
%log is operating correctly (i.e., including the certificate in its Merkls hash
%tree after issuing \iac{sct}), misbehavior \emph{should} be detected at some
%point.

\subsection{Policy Operations}

\draft{When a log receives a new certificate chain, it checks the domain name
  and adds the chain to the set of chains for the appropriate public key in its
  map, and then recomputes the strength of this new set of certificate chains.
  If the maximum number strength associated with a public key for the domain has
increased, it updates its policy tree accordingly.}

\draft{When a certificate expires, the number for the public key in that
certificate is decremented. If the certificate is renewed before expiration
(i.e., the same \ac{ca} issues a new certificate for the same key), then the
number for that key is not incremented when the renewal certificate is logged,
and the number is not decremented when the old certificate expires.}

\draft{If a certificate is revoked, the number is also decremented.}

\draft{If a domain wants to change \acp{ca} for a certain key upon expiration or
  revocation of a certificate containing that key, the domain can either wait
  until expiration or revocation to obtain a new certificate or explicitly
  signal to the new \ac{ca} that it is changing \acp{ca}. The new \ac{ca} can
then include in its precertificate a dummy extension (i.e., an extension marked
critical but with no data, so that a client will always reject the certificate)
that directs the log to not increment the number for that key.}

%The initial policy registration can be performed when the log has no entry for
%the given domain. A log performs the initial policy registration when its
%receives a certificate whose subject name does not match any domain that the log
%has in its mapping. In this case, the log simply creates a new policy entry for
%that domain and associates the submitted certificate's public key with the
%domain. The policy value is set to be the default value of 1 (indicating that
%a client should expect only a single logged certificate chain from the domain).

%Each new public key or certificate needs to be bound to a policy ID as well as
%to the domain itself. It should be impossible to bind a new key to a policy ID
%without the approval of the policy holder. Thus the domain will need to provide
%some additional information regarding its authorization of the addition or
%removal of any public key or certificate.

%We therefore want the policy ID to be stable or at least linkable given an
%addition or deletion operation. We also want the ID to be used to authenticate
%a signature authorizing the addition or deletion of a key.

%\subsection{Policy Updates}

%\steve{Policies can be strengthened (i.e., not updated in terms of content but
%in terms of independent chains) by simply specifying the same policy.}

%\subsection{Policy Overwriting}

%\subsection{Log Policy Proofs}

%\steve{The irrelevance of log inclusion or consistency proofs to clients does
%not hold in \ac{name}. This is because clients need to be convinced that a
%policy was logged and that no policy overwrites or updates have taken place
%since (i.e., the client sees the latest version of the policy). (Actually, if
%the policy was logged and is not the latest, that could still prevent many
%\acp{ca} from issuing unauthorized certificates if using the authorized \ac{ca}
%approach. If we use the number of independent chains approach, we could include
%a Bloom filter--like structure that makes it difficult for an adversary to get
%away with fewer certificates even by forging them. How secure are Bloom filters
%to collision attacks, anyway?)}

%\subsection{Other Operations}

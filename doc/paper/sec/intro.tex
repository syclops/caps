\section{Introduction}
\label{sec:intro}

\acs{https} is fundamental for secure Web communication. When a user Alice
wishes to securely access Bob's site \texttt{bob.com}, \ac{https} allows Bob to
serve his site over a secure communication channel that provides secrecy and
integrity. To establish this channel, Alice and Bob perform the \ac{tls}
handshake protocol~\cite{rfc8446}, which allows Bob to use his public key
$\pk_B$ to establish a shared secret key with Alice, which they can subsequently
use for encrypted communication.

To convince Alice that $\pk_B$ should be associated with \texttt{bob.com},
however, \ac{https} relies on the Web \ac{pki}. A trusted third party called a
\emph{\ac{ca}} checks that Bob owns both \texttt{bob.com} and the private key
corresponding to $\pk_B$, and
issues a digitally signed certificate that vouches for this binding. \acp{ca}
thus play a crucial role in secure Web communication: the failure of any \ac{ca}
due to error, compromise, or coercion can lead to a certificate that binds
\texttt{bob.com} to a different public key $\pk_M$. If for example this key
belongs to an adversary Mallory, she can impersonate Bob to Alice in a
\emph{\ac{mitm} attack}, one of the main problems that \iac{pki} aims to solve.

Unfortunately, the current Web \ac{pki} is demonstrably fragile. Existing
certificate databases indicate that Web browsers and operating systems provided
by Mozilla, Apple, and Microsoft directly or indirectly trust more than 1,500
\ac{ca} signing keys across more than 600 organizations worldwide. There are few
measures in place to prevent any of these \acp{ca} from issuing an unauthorized
certificate for any site, resulting in \emph{weakest-link security} for most
sites: the compromise of any \ac{ca} can threaten the security of all Web
domains, and by extension, all clients visiting sites on those domains. Recent
years have seen a plethora of incidents where misbehaving \acp{ca} issued
unauthorized certificates, both accidentally~\cite{sleevi2015sustaining} and
intentionally~\cite{valsorda2015komodia}.

As we describe in \autoref{sec:background}, previous work has made progress
towards protecting clients and domains against the misbehavior of trusted
entities such as \acp{ca}, and some of this work has seen increasingly
widespread deployment. Unfortunately, no proposed solution offers both
preemptive, robust protection against misbehaving \acp{ca} and a feasible,
secure deployment strategy. In particular, while systems like Google's \ac{ct}
project~\cite{rfc6962,ct-laurie} enjoy relatively widespread deployment, they only enable
detection, not prevention, of unauthorized certificate issuance. Systems that do
prevent unauthorized certificate issuance either add new trusted
parties~\cite{kim2013accountable}, require domains to deploy complex new
infrastructure~\cite{rfc6698, szalachowski2014policert}, significantly increase
latency and communication~\cite{yu2016dtki}, or require all domains to increase
their security level at once~\cite{basin2014arpki}. Many of these systems also
require \acp{ca} to drastically change their operational processes or require
domains to undergo a complex and error-prone deployment process in which
mistakes are difficult to fix. For example, public-key pinning~\cite{rfc7469,mid-air}
resulted in many misconfigurations that rendered domains inaccessible; as a result,
Google recently decided to remove support for it~\cite{palmer2017intent}.

To address these shortcomings, we propose \ac{name}, a system that provides a
modest roadmap for transitioning to a more resilient Web \ac{pki}. In \ac{name},
domains can take simple steps to protect themselves from \ac{mitm} attacks in
the presence of one or more misbehaving \acp{ca}. Furthermore, \ac{name} offers
an incremental  deployment strategy free of the perils experienced in previous
work: the interaction between domains and \acp{ca} remains fundamentally the
same, and clients are protected from downgrade attacks such as \ac{tls}
stripping~\cite{marlinspike2009new} during deployment. In \ac{name}, the
existing \ac{pki} can coexist with one that offers stronger security guarantees,
allowing domains to decide when they are ready to improve their own security
without penalizing non-deployers. The bulk of the deployment effort of \ac{name}
can occur at a small handful of participants, namely browser vendors and public
logs, who are better equipped to make these changes than domains or \acp{ca}
that have historically been reluctant to deploy major changes.

\ac{name} achieves the above properties through several important observations.
First, because the Web \ac{pki} encompasses a vast set of domains with diverse
security needs, few changes to the \ac{pki} will be universally adopted
overnight. Thus we must have a \emph{signaling mechanism} to indicate which
domains have adopted a new \ac{pki} enhancement. \ac{name} conveys this signal
with the use of a single bit that indicates whether a domain supports
\ac{https}. Without this bit \ac{tls} stripping~\cite{marlinspike2009new} can
occur, causing the client to ignore even the existing \ac{pki}. With the use of
this bit, the remaining signaling information can be communicated as an
extension to the \ac{tls} handshake. Because existing solutions for indicating
\ac{https} deployment require the client to trust its first connection to the
domain, they have lagged in deployment~\cite{rfc4033, rfc6698}, and add
significant storage overhead\footnote{\url{https://www.eff.org/https-everywhere}}
or connection
latency.\footnote{\url{https://mybrowseraddon.com/smart-https.html}} Instead, we
use data compression techniques and compact data structures to locally store at
each client a succinct summary of the \emph{entire} set of domains that deploy
\ac{https}; with our prototype, it requires less than 140~MB of storage
(and Section~\ref{sec:evaluation:implementation} discusses additional options 
to reduce this overhead).

We also observe that all known instances of \ac{ca} misbehavior up to this point
have involved a single misbehaving \ac{ca}.  Hence, \ac{name} allows a domain to
authenticate its public key using multiple independent certificate chains (we
quantify the impact this has on connection latency and bandwidth in
Section~\ref{sec:evaluation}). \ac{name}-enabled clients know how many chains to
expect for each domain, and hence, the simultaneous failure of $n-1$ \acp{ca}
cannot enable \iac{mitm} attack on a domain with $n$ chains. Note that \ac{name}
signals to clients the \emph{number} of certificate chains to expect, rather
than the list of authorized \acp{ca}.  This results in a useful practical
property: misconfiguration or private key loss cannot render a domain
inaccessible (a known problem in some prior proposals), 
since the domain can simply obtain the requisite number of
certificates to establish a new authoritative public key.

To vouch for the number of certificate chains that a client should expect to
receive, \ac{name} uses publicly auditable authenticated database services based
on a global view of all certificates in the Web \ac{pki}, which allows it to
\emph{automatically} infer the number of certificate chains for each domain.
Previous work has shown that such a global view can be valuable in understanding
and improving the Web \ac{pki}~\cite{durumeric2015search, larisch2017crlite}.
% \steve{Cite the certificate linting paper too}

Finally, we observe that a strongly authenticated, authoritative public key
for a domain (as provided by \ac{name}) is a powerful force: it can be used on its own to provide greater
confidence in a site's identity, or it can be used to authenticate richer
policies as proposed in prior work~\cite{basin2014arpki,
szalachowski2014policert}. \ac{name} can thus simplify the deployment of these
proposed systems, whose existing deployment and certificate issuance strategies
rely on complex coordination among domains, \acp{ca}, and public logs to certify
these policies. \ac{name} also enhances the recoverability of these systems,
which, as originally proposed, require waiting for days to replace a policy if
the corresponding private key is lost or compromised.

Even without considering wholly new \acp{pki}, \ac{name} is attractive from a
deployment standpoint. In particular, the administrative burden of deployment
for domains is limited to acquiring additional certificates, and with the use
of free certificate services like Let's
Encrypt,\footnote{\url{https://letsencrypt.org/}} the financial burden can be
minimized as well. Furthermore, \ac{name} is an opt-in system, meaning that only
domains who choose to obtain additional certificates incur a cost. From its
initial deployment, \ac{name} protects all domains from \ac{tls} stripping, and
it allows non-deploying domains to coexist with deploying domains without
enabling downgrade attacks on deploying domains.

In summary, we make the following contributions.
\begin{compactitem}
\item We present the design of \ac{name}, which provides an incrementally deployable,
      backwards compatible path to a more resilient Web PKI.
\item We show how a security policy based on the number of independent certificate chains
      a domain can obtain strikes a good balance between automation, 
      resilience to attack, and robustness to domain errors (including private key loss).
\item We combine a global view of Web certificates with techniques from data
      compression and compact data structures to succinctly signal \ac{https} deployment
\item We demonstrate via an evaluation of a prototype that the client-side overhead for
      \ac{name} in terms of storage, memory, and connection-establishment latency is
      modest. 
\end{compactitem}


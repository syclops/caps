\section{Introduction}
\label{sec:intro}

%Some questions to think about:
%\begin{compactitem}
%\item How can we move toward a world in which domains use multiple independent
  %certificate chains?
%\item How do we validate a self-signed certificate?
%\item What does it mean to have trust agility?
%\item Does the use of multiple independent certificate chains provide better
  %security and agility?
%\end{compactitem}

\acs{https} is fundamental for secure Web communication. When a user Alice
wishes to securely access Bob's site \texttt{bob.com}, \ac{https} allows Bob to
serve his site over a secure communication channel that provides secrecy and
integrity. To establish this channel, Alice and Bob perform the \ac{tls}
handshake protocol~\cite{rfc5246}, which allows Bob to use his public key
$\pk_B$ to establish a shared secret key with Alice, which they can subsequently
use for encrypted communication.

To convince Alice that $\pk_B$ should be associated with
\texttt{bob.com}, however, \ac{https} relies on the Web \ac{pki}. Trusted
third parties called \emph{\acp{ca}} can check that Bob owns both
\texttt{bob.com} and $\pk_B$, and issue a digitally signed certificate that
binds this information. \acp{ca} thus play a crucial role in secure Web
communication: the failure of any \ac{ca} to do its due diligence in checking
Bob's key can lead to a certificate that binds \texttt{bob.com} to a
different public key $\pk_M$. If for example this key belongs to an adversary
Mallory, she can impersonate Bob to Alice in a \emph{\ac{mitm} attack}.

Unfortunately, the current Web \ac{pki} is demonstrably fragile. Existing
certificate databases indicate that more than 1,500 \acp{ca} trusted by Mozilla,
Apple and Microsoft in their respective Web browsers and operating systems.
There are few measures in place to prevent any of these \acp{ca} from issuing a
certificate for any site, resulting in \emph{weakest-link security} for most
sites: the compromise of any \ac{ca} can threaten the security of all Web
domains, and by extension, all clients visiting sites on those domains. Recent
years have seen a plethora of incidents where misbehaving \acp{ca} issued
unauthorized certificates, both accidentally~\cite{sleevi2015sustaining} and
intentionally~\cite{valsorda2015komodia}.

While previous work has resulted in systems that aim to make the \ac{pki} more
resilient to the effects of misbehaving \acp{ca}, these advances have been
insufficient to create a truly secure Web \ac{pki}. Systems like Google's
\ac{ct} project~\cite{rfc6962} only allow for detection, not prevention, of
unauthorized certificates. Systems that do prevent clients from unauthorized
certificates add new trusted parties~\cite{kim2013accountable}, add complex
new infrastructure~\cite{szalachowski2014policert}, drastically increase latency
and communication~\cite{yu2016dtki}, or require a flag day deployment where all
domains increase their security level at once~\cite{basin2014arpki}. Many of
these systems also require domains to put significant effort into creating
complex policies, a process that has proven error-prone even in today's limited
Web \ac{pki} improvements such as public-key pinning~\cite{palmer2017intent}.

To address these shortcomings, we propose \acs{name}, a system that provides a
modest roadmap to transition to a more resilient Web \ac{pki}. Specifically,
\ac{name} proposes \iac{pki} that prevents unauthorized certificates issued by
individual misbehaving \acp{ca} from being accepted by clients, and allows this
\ac{pki} to coexist with the current Web \ac{pki}. By enabling this overlap of
\acp{pki} and by preventing downgrade attacks between them, \ac{name} allows
domains to decide when they are ready to transition to a new \ac{pki}, avoiding
performance penalties for non-deployers while providing stronger security
guarantees to even the earliest adopters. \ac{name} also focuses its major
changes on a small handful of participants, namely browser vendors and public
logs, who are better equipped to make these changes than the millions of domains
that currently deploy \ac{https}. By minimizing the changes at the domain, we
also avoid the perils of domain misconfiguration that domains have faced with
previous systems.

\ac{name} achieves the above goals through several important observations.
First, because the Web \ac{pki} encompasses a vast set of domains with diverse
security needs, few changes to the \ac{pki} will be universally adopted
overnight. Thus we must have a \emph{signaling mechanism} to indicate which
domains have adopted a new \ac{pki}. Moreover, this signal must be extrinsic to
the \ac{pki} itself, since otherwise a client cannot distinguish a non-deploying
domain from an attacker attempting to subvert the connection through a downgrade
attack.

We further note that we can convey this extrinsic signal with the use of a
single bit that indicates whether a domain has deployed \ac{https}. Without this
bit \ac{tls} stripping can occur, causing the client to ignore even the existing
\ac{pki}, while with this bit, the remaining signaling information can be moved
into the \ac{tls} handshake as an extension. Because existing solutions for
indicating \ac{https} deployment require trusting the first connection to the
domain, have lagged in deployment~\cite{rfc4033}, and add significant storage
overhead \steve{HTTPS Everywhere} or connection latency \steve{Smart HTTPS}, we
use data compression techniques and compact data structures to locally store the
\emph{entire} set of domains that deploy \ac{https}.

We also observe that all known instances of \ac{ca} misbehavior up to this point
have involved a single misbehaving \ac{ca}. Thus we must prevent the misbehavior
of individual \acp{ca} from enabling \ac{mitm} attacks by enabling domains to
authenticate their public keys using multiple \ac{ca} signatures. Moreover, this
process must prevent misconfiguration or private key loss from rendering a
domain inaccessible as in existing systems for preventing \ac{mitm} attacks. We
note that by having domains send multiple certificate chains authenticating a
single public key and by enabling domains to simply communicate the number of
chains a client should expect, we can protect domains from single misbehaving
\acp{ca} and signal the deployment of a more resilient \ac{pki} without exposing
domains to practical hazards. Domains can recover from misconfiguration or
private key loss by simply obtaining certificates on a new public key.

We note that Google's \ac{ct} system is now deployed in all Chrome browsers and
required for all \ac{https}-enabled sites to be shown as secure in these
browsers. Therefore, the public log system in \ac{ct} provides a global view of
all certificates in the Web \ac{pki}. This global view allows logs to
\emph{automatically} infer the number of expected certificate chains for each
domain, which \ac{name} can leverage to create an authenticated database for
this information. This database allows domains to establish an authoritative
public key backed by certificates from multiple \acp{ca}. With this approach,
domains can set their own desired level of resilience to compromised \acp{ca},
and do so by interacting with \acp{ca} in the same way they do today, simply
purchasing certificates from them.

Finally, we observe that once we can establish a strongly authenticated,
authoritative public key for a domain, we can use this key to secure richer
policies as proposed in prior work, such as a list of trusted \acp{ca}. As is
desirable for securing a high-value policy, this authoritative key cannot be
overruled by a single misbehaving \ac{ca}, provides a domain-adjustable level of
security, and does not lock out domains in case of misconfiguration or private
key loss. Even if a domain's private key is compromised and the adversary gains
additional certificates, the domain can get these certificates revoked.

\ac{name} is also attractive from a deployment standpoint. In particular, the
burden of deployment for domains is limited to a simple server-side patch and
the cost of additional certificates for domains that choose to obtain them. The
bulk of changes focus on browser vendors and \ac{ct} logs, who are well equipped
to make such changes and few enough that it is feasible to convince a majority
of them to make the necessary changes. \ac{name} also protects all domains from
\ac{tls} stripping from its initial deployment, even non-deploying domains, and
allows non-deploying domains to coexist with deploying domains without enabling
downgrade attacks.

In summary, we make the following contributions:
\begin{compactitem}
\item We conduct a study of the existing \ac{pki} to assess the current state of
  \ac{https} among domains and the expected growth of \ac{https} deployment.
\item We leverage techniques from data compression and compact data structures
  to present a design for locally storing the entire set of \ac{https}-deploying
  domains on the Web.
\item We extend \ac{ct} logs to present a simple, deployable, and automated
  mechanism for maintaining and disseminating domain certificate policies.
\item We evaluate \ac{name} against existing approaches and shows that
  \ac{name} outperforms these approaches, requiring only \steve{} extra
  communication, \steve{} additional storage, and \steve{} additional latency.
\end{compactitem}


%So far, proposed systems for communicating policy information to clients have
%not gained traction in practice. These systems introduce additional trusted
%parties~\cite{kim2013accountable} or assume that the entire Web will increase
%its security at once~\cite{basin2014arpki}. These shortcomings motivate the
%problem of designing a simple and deployable policy mechanism that requires
%minimal effort from domains. \steve{The last part isn't motivated well; it would
  %be nice if we could show that most policies are overkill for preventing
%\ac{mitm} attacks that could have resulted from recent \ac{ca} failures.}


%The Web's \ac{pki} enables client browsers to authenticate the public keys of
%the domains they communicate with, and together with the \ac{tls}
%protocol~\cite{rfc5246} allows a client and domain to establish an end-to-end
%secure communication channel through \acs{https}. In the Web \ac{pki},
%trusted third parties called \emph{\acp{ca}} vouch for the binding between a
%domain's name and its public key. The correct operation of \acp{ca} are critical
%to the security of the Web \ac{pki}, and failures can lead to the client
%establishing a secure channel with the wrong domain. In some cases this
%scenario can lead to a \emph{\ac{mitm} attack}, in which even though the
%confidentiality and integrity guarantees of communication over \ac{tls} hold,
%they provide no benefit because an adversary is impersonating the client and
%domain to each other and can read and modify all communication between them.

%While there are many proposals (enumerated in \autoref{sec:related}) to fix the
%Web \ac{pki} by deterring, preventing, or recovering from \ac{ca} misbehavior,
%Google's \emph{\ac{ct}} system~\cite{rfc6962} has recently emerged as a widely
%deployed example of such a system, being scheduled for full deployment in
%October 2017. \ac{ct} does not prevent \ac{mitm} attacks, but rather enables
%widespread detection of \ac{ca} misbehavior. In \ac{ct}, a set of \emph{public
%logs} maintain a database of certificates they have observed (primarily sent to
%them by the \acp{ca} themselves) and provide cryptographic proofs called
%\emph{\acp{sct}} that show when these certificates were submitted to the log.
%Clients do not accept certificates unless they are sent with a requisite number
%of these \acp{sct}, ensuring that an adversary cannot use an unauthorized
%certificate for \ac{mitm} attacks without first exposing that certificate to the
%world.

%However, \ac{ct} has several notable shortcomings. \ac{ct} does not make
%\ac{mitm} attack prevention an explicit goal, but specifies the behavior of
%\emph{monitors}, who periodically download newly logged certificates and check
%for suspicious certificates. However, only domains know which certificates they
%have authorized, and \ac{ct} does not allow domains to check logs for
%certificates issued in their name. This disparity means that some manual effort,
%on the part of the monitors, domains, or both, is required to prevent \ac{mitm}
%attacks that result from unauthorized certificates. Both monitors and domains
%face hurdles due to this manual effort.

%With the increasing deployment of \ac{https} through certificate services such
%as Let's Encrypt,\footnote{\texttt{https://letsencrypt.org/}} monitors may not
%be able to keep up with the number of suspicious certificates that require
%further manual investigation. Previous work enables domains to alleviate this
%pressure on monitors by explicitly supplying \emph{policies} that specify
%criteria that a domain's certificates should meet, but it is not clear that the
%expressiveness of these policies and the burden they impose on domain operators
%is worth the benefit of protecting them from common certificate misissuances. We
%therefore seek a simpler, more automated system for specifying certificate
%policies through the \ac{ct} infrastructure. In particular, we aim to move
%towards \iac{pki} in which one-off mistakes by \acp{ca} do not expose domains to
%\ac{mitm} attacks.

%Unfortunately, even with such policies, we still fall short of a truly secure
%\ac{pki}. Domains may not only have different policies, but some domains may not
%have policies at all because they do not deploy \ac{https}. Therefore, an
%adversary can carry out a \emph{\ac{tls} stripping} attack, in which the
%adversary convinces the client and domain to communicate over plain HTTP rather
%than over \ac{https}. Because we realistically expect a world of partial
%deployment in which the set of deploying domains increases incrementally over
%time, in order to prevent \ac{tls} stripping, we need a mechanism to inform
%clients \emph{when} to expect a certificate policy, not just \emph{what} policy
%to expect. We thus consider the following problem, which we call
%\emph{\ac{https} signaling}: how can we communicate to all clients the set of
%domains from which they should expect \iac{tls} certificate and certificate
%policy?

%Previous proposals (\autoref{sec:background:signaling}) have addressed this
%problem by requiring expensive online checks \steve{Smart HTTPS}, requiring
%domains to deploy DNSSEC~\cite{rfc6698}, or providing an incomplete view of the
%domains signaling \ac{https} deployment \steve{HTTPS Everywhere, HSTS
%preloading}. These approaches incur significant latency, are
%configuration-intensive and error-prone, or have limited scope. Moreover, we
%observe that signaling mechanisms must operate externally to the Web \ac{pki}:
%approaches such as Smart HTTPS or HSTS rely on certification within the Web
%\ac{pki} for a client's first connection to a domain and thus are themselves
%subject to stripping attacks. We therefore seek a solution that captures the set
%of all \ac{https}-enabled domains, incurs minimal additional latency, is simple
%to deploy, and is external to the Web \ac{pki}.

%%Existing approaches to the signaling problem
%%(\autoref{sec:background:signaling}) suffer from critical shortcomings. Some
%%require domains to opt in using a process prone to misconfiguration, and render
%%the domain inaccessible if misconfiguration occurs. Approaches that store
%%deployment information locally only signal \ac{https} deployment in a subset of
%%sites due to storage costs, and approaches that rely on external information
%%incur significant latency penalties and introduce additional trusted parties.
%%We therefore observe the need to address the signaling problem with a solution
%%that allows clients to confidently distinguish a site that deploys \ac{https}
%%without imposing performance penalties or trust assumptions.

%To address the problem of moving towards a Web \ac{pki} more resilient to the
%compromise of trusted parties, we thus tackle the certificate policy
%specification and \ac{https} signaling problems. Our solution to these problems
%is a system called \ac{name}, whic gradually increases the resilience of the Web
%\ac{pki} against misbehaving \acp{ca}.

%\ac{name} leverages Google's \ac{ct}~\cite{rfc6698} infrastructure
%(\autoref{sec:background:ct}), which relies on \emph{public logs} that maintain
%a database of all issued certificates in the Web \ac{pki}. We observe that these
%databases can be used to signal the set of all domains that deploy \ac{https}.
%By leveraging compression techniques and knowledge of domain name patterns and
%trends, we can communicate a single bit indicating \ac{https} deployment for
%each domain. Furthermore, domains who deploy \ac{https} can be automatically
%detected, because their certificates will be included in these log databases.

%Once a client can determine whether or not to expect a certificate policy
%through the use of \ac{name}'s signaling mechanism, we can communicate simple
%policies that require minimal domain involvement while preventing \ac{mitm}
%attacks resulting from most cases of \ac{ca} misbehavior. In particular,
%\ac{name} extends \ac{ct} logs to also maintain certificate policies. We can
%achieve this by allowing domains to send multiple certificate chains for a
%single public key during the \ac{tls} handshake, and then using certificate
%policies to communicate to clients how many chains they should expect. This
%approach allows logs to infer the policy's value based on the certificates that
%are already logged in its database, and also prevents certificate misissuances
%by individual misbehaving \ac{ca} from enabling \ac{mitm} attacks against these
%domains.

%Because all certificates must be publicly logged to be accepted in Chrome
%browsers starting in October
%2017,\footnote{\url{https://cabforum.org/2016/10/19/2016-10-19-20-f2f-meeting-39-minutes/}}

%The Web \ac{pki} has seen several important improvements in recent years.
%\steve{Summary of CT, CAA, DANE, CAge, CRLite}

%Because \acp{ca} are trusted parties in the \ac{tls} \ac{pki}, they can enable
%\ac{mitm} attacks if they misbehave. Unfortunately, they have done so multiple
%times over the years, due to social engineering
%attacks~\cite{microsoft2001erroneous}, network breaches~\cite{comodo2011fraud,
  %hoogstraaten2012black}, operational errors~\cite{zusman2009sub,
  %langley2013enhancing, somogyi2015improved, sleevi2015sustaining}, or malicious
  %intent~\cite{percoco2012clarifying, langley2013further,
  %langley2015maintaining}. While in many cases the rogue certificates were
  %quickly detected and revoked, in some cases the certificates were actually
  %used in the wild in \ac{mitm} attacks~\cite{nightingale2011diginotar}. These
  %attacks are particularly noxious because the victim website may never find out
  %that it is being impersonated. Indeed, we have seen instances of targeted
  %attacks using fake certificates to impersonate websites to specific groups of
  %users~\cite{eckersley2011syrian}.

%While many attempts have been made to detect and prevent \ac{ca} misbehavior,
%one of the most widely-deployed fixes is \ac{ct}~\cite{rfc6962}. The goal of
%\ac{ct} is to enable fast and widespread detection of certificate issuance by
%publicly logging all certificates in \ac{tls}. A set of \emph{public logs}
%record all issued certificates in an append-only, tamper-proof fashion, and
%clients only accept a \ac{tls} certificate if it has been logged, thus forcing a
%potential adversary to publicize a certificate (making it detectable) before
%mounting a \ac{mitm} attack. However, simply adding logs to the existing
%\ac{tls} has several critical shortcomings. In particular, we identify three
%flaws with the current log-based architecture.


%a domain is associated with a policyin order to connect securely, a client needs
%to make sure that
%the policy was logged and registered (presence)
%the policy has not been updated since the value being read (recency)
%the policy was registered by the real domain (authenticity)

%what is the bad thing of having an incorrect policy?
  %accept too few certs or certs signed by the wrong ca (aka atker controlled)

%authenticated key-value store


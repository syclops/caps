\section{Introduction}
\label{sec:intro}

\acs{https} is fundamental for secure Web communication. When a user Alice
wishes to securely access Bob's site \texttt{bob.com}, \ac{https} allows Bob to
serve his site over a secure communication channel that provides secrecy and
integrity. To establish this channel, Alice and Bob perform the \ac{tls}
handshake protocol~\cite{rfc8446}, which allows Bob to use his public key
$\pk_B$ to establish a shared secret key with Alice, which they can subsequently
use for encrypted communication.

To convince Alice that $\pk_B$ should be associated with \texttt{bob.com},
however, \ac{https} relies on the Web \ac{pki}. A trusted third party called a
\emph{\ac{ca}} checks that Bob owns both \texttt{bob.com} and the private key
corresponding to $\pk_B$, and
issues a digitally signed certificate that vouches for this binding. \acp{ca}
thus play a crucial role in secure Web communication: the failure of any \ac{ca}
due to error, compromise, or coercion can lead to a certificate that binds
\texttt{bob.com} to a different public key $\pk_M$. If for example this key
belongs to an adversary Mallory, she can impersonate Bob to Alice in a
\emph{\ac{mitm} attack}, one of the main problems that \iac{pki} aims to solve.

Unfortunately, the current Web \ac{pki} is demonstrably fragile. Existing
certificate databases~\cite{durumeric2015search, rfc6962} indicate that Web
browsers and operating systems provided by Mozilla, Apple, and Microsoft
directly or indirectly trust more than 1,500 \ac{ca} signing keys across more
than 600 organizations worldwide. There are few measures in place to prevent any
of these \acp{ca} from issuing an unauthorized certificate for any site,
resulting in \emph{weakest-link security} for most sites: the compromise of any
\ac{ca} can threaten the security of all Web domains, and by extension, all
clients visiting sites on those domains. Recent years have seen a plethora of
incidents where misbehaving \acp{ca} issued unauthorized certificates, both
accidentally~\cite{sleevi2015sustaining} and
intentionally~\cite{valsorda2015komodia}.

As we describe in \autoref{sec:background}, previous work has made progress
towards protecting clients and domains against the misbehavior of trusted
entities such as \acp{ca}, and some of this work has seen increasingly
widespread deployment. Unfortunately, no proposed solution offers both
preemptive, robust protection against misbehaving \acp{ca} and a feasible,
secure deployment strategy. In particular, while systems like Google's \ac{ct}
project~\cite{rfc6962,ct-laurie} enjoy relatively widespread deployment, they
only enable detection, not prevention, of unauthorized certificate issuance.
Systems that do prevent unauthorized certificate issuance will
\begin{inparaenum}
\item require drastic changes to certificate issuance~\cite{kim2013accountable},
\item require domains to deploy complex new infrastructure~\cite{rfc6698,
  szalachowski2014policert},
\item significantly increase latency and communication~\cite{yu2016dtki},
\item require all domains to increase their security level at
  once~\cite{basin2014arpki}, or
\item render a domain inaccessible if misconfigured~\cite{palmer2017intent}.
\end{inparaenum}

To address these shortcomings, we propose \ac{name}, a system that provides a
modest roadmap for transitioning to a more resilient Web \ac{pki}. Compared to
previous work, \ac{name} provides similar resilience against \ac{mitm} attacks
while also providing an improved incremental deployment strategy. Interactions
between domains and \acp{ca} remain unchanged in the vast majority of cases, and
simple misconfigurations do not cause domains to become inaccessible. \ac{name}
provides immediate benefits to domains and clients who deploy it, but does so in
a way that does not penalize non-deploying domains. \ac{name} also protects
clients from downgrade attacks such as \emph{\ac{tls}
stripping}~\cite{marlinspike2009new} during deployment. Finally, the bulk of the
deployment effort of \ac{name} can occur at a small handful of participants,
namely browser vendors and public logs, who are better equipped to make these
changes than domains or \acp{ca} that have historically been reluctant to deploy
major changes.

The design of \ac{name} relies on several important observations. First,
\emph{our near-complete view of the modern Web \ac{pki} can be used as a channel
for domains to communicate information}. Fast, Internet-wide
scanning~\cite{durumeric2013zmap} and browser-based Web \ac{pki}
standards~\cite{sleevi2016requiring} have resulted in large databases of
\ac{tls} certificates~\cite{durumeric2015search}, and these databases have
already been used for insights~\cite{vandersloot2016towards} and
improvements~\cite{larisch2017crlite} in the Web \ac{pki}. \ac{name} can use
this view of the Web \ac{pki} to obtain domain information, including \ac{ca}
information and public keys, without domains having to take any additional action
beyond obtaining a certificate (which they already do). Therefore, \ac{name}
leverages these databases to free domains and \acp{ca} from further changes to
their operational procedures.

Second, \emph{a viable deployment strategy must signal both deployment of
\ac{https} and deployment of \ac{pki} enhancements to clients}. The Web \ac{pki}
encompasses a vast set of domains with diverse security needs, and it is highly
unlikely that any change to the \ac{pki} will be universally adopted overnight.
Therefore \ac{name} provides a \emph{signaling mechanism} that conveys
\begin{inparaenum}
\item whether a domain supports \ac{https}, and
\item if so, whether it supports \ac{name}.
\end{inparaenum}
Without the first bit of information, \ac{tls} stripping can occur, causing the
client to ignore even the existing \ac{pki}. With this bit, the remaining
signaling information can be communicated as an extension to the standard
\ac{tls} handshake. Previous approaches have required operational changes to
domains~\cite{rfc4033, rfc6698} or significant storage overhead
(\autoref{sec:background:https}). \ac{name} uses a global view of the Web
\ac{pki} in conjunction with data compression techniques and compact data
structures to locally store at each client a succinct summary of nearly the
\emph{entire} set of domains that deploy \ac{https}. In our prototype, we
summarize over 64 million domain names with less than 150~MB of storage (and
Section~\ref{sec:evaluation:implementation} discusses additional tradeoffs that
can reduce this overhead to less than 35~MB).

Third, \emph{any domain that supports \ac{https} has obtained a certificate
using an existing issuance process, and can use this process to obtain
certificates from further \acp{ca}}. Domains in \ac{name} can thus use multiple
independent certificate chains for the same public key to establish one or more
``authoritative'' public keys (\autoref{sec:design:policy}). By communicating
only the \emph{number} of independent chains that denote an authoritative key,
\ac{name} makes it easy for a domain to recover from the loss or compromise of a
private key or from a misconfiguration. The domain simply obtains an equal or
greater number of chains for a new public key to establish it as authoritative.
Moreover, the global view we have of the Web \ac{pki} allows \ac{name} to
\emph{automatically} keep this number up to date with no changes to current
certificate issuance processes and no additional action from domains. Even with
this flexibility, authoritative public keys in \ac{name} are hard to forge: if a
public key is backed by $n$ independent chains, an adversary must forge \emph{at
least} $n$ certificates for the same public key to carry out \iac{mitm} attack.

Finally, \emph{authoritative public keys benefit both current and future Web
\acp{pki}}. These keys can be used on their own to provide greater confidence in
a site's identity, or they can be used to authenticate richer policies as
proposed in prior work~\cite{basin2014arpki, szalachowski2014policert}.
\ac{name} can thus simplify the deployment of these proposed systems, whose
existing deployment and certificate issuance strategies rely on complex
coordination among domains, \acp{ca}, and public logs to certify these policies.
\ac{name} also enhances the recoverability of these systems, which, as
originally proposed, require waiting for days to replace a policy if the
corresponding private key is lost or compromised.

Even without considering wholly new \acp{pki}, \ac{name} is attractive from a
deployment standpoint. In particular, the administrative burden of deployment
for domains is limited to acquiring additional certificates, and with the use of
free certificate services like Let's
Encrypt,\footnote{\url{https://letsencrypt.org/}} the financial burden can be
minimized as well. Furthermore, \ac{name} is an opt-in system, meaning that only
domains who choose to obtain additional certificates incur a cost, and this cost
is purely for purchasing the certificates. From its initial deployment,
\ac{name} protects all domains from \ac{tls} stripping, and it allows
non-deploying domains to coexist with deploying domains without enabling
downgrade attacks on deploying domains.

In summary, we make the following contributions.
\begin{compactitem}
\item We present the design of \ac{name}, which provides an incrementally deployable,
      backwards compatible path to a more resilient Web PKI.
\item We show how a security policy based on the number of independent certificate chains
      a domain can obtain strikes a good balance between automation, 
      resilience to attack, and robustness to domain errors (including private key loss).
\item We combine a global view of Web certificates with techniques from data
      compression and compact data structures to succinctly signal \ac{https} deployment
\item We demonstrate via an evaluation of a prototype that the client-side overhead for
      \ac{name} in terms of storage, memory, and connection-establishment latency is
      modest. 
\end{compactitem}


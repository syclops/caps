\section{Background and Related Work}
\label{sec:background}

We provide a brief overview of the work related to the three problems we address
in this paper: tracking Web certificates, enforcing the use of \ac{https}, and
improving the certificate issuance and validation processes.

\subsection{Tracking Web Certificates}\label{sec:tracking}

Censys is a service that uses ZMap~\cite{durumeric2013zmap} to provide a search
engine of network devices and infrastructure in the IPv4 address
space~\cite{durumeric2015search}. In the context of the Web \ac{pki}, Censys
keeps records of periodic \ac{tls} handshake attempts to the entirety of the
public IPv4 address space on port 443 (the standard port for \ac{https}) dating
back to 2015. Censys provides a database of information on certificates received
in these handshakes such as validation in different operating systems.

\acf{ct} publicizes certificate issuance~\cite{rfc6962}. \ac{ct} introduces the
role of \emph{certificate logs} to the Web \ac{pki}, entities that use Merkle
hash trees~\cite{merkle1988digital} to maintain an auditable, append-only, and
tamper-proof database~\cite{crosby2009efficient} of certificates in the Web
\ac{pki}. A domain obtains proofs from logs that its certificate has been
logged, and sends these proofs to clients during the TLS handshake inside a
\ac{tls} extension.\footnote{Two extensions may be used: \iac{ct}-specific
extension or the Certificate Status Request extension, also known as OCSP
stapling~\cite{rfc6066}} CT-enabled clients reject certificates that are not
accompanied by such a proof.

Censys and \ac{ct} can be used in conjunction to provide a reasonably complete
view of the Web's \ac{https} ecosystem, particularly the \acp{fqdn} of sites
deploying HTTPS~\cite{vandersloot2016towards}. Previous work has used this view
to efficiently represent the set of all revoked
certificates~\cite{larisch2017crlite}.

\subsection{Enforcing \ac{https}}
\label{sec:background:https}

\ac{hsts} is a mechanism that allows sites to tell a client connecting over HTTP
that all future connections to the site should take place over
\ac{https}~\cite{rfc6797}. A site typically does this by redirecting to
\ac{https} and providing a \emph{strict transport security policy} in its
response specifying the time period and subdomains for which the client should
use \ac{https}. \ac{hsts} is \emph{\ac{tofu}}, meaning that the adversary can
only mount \iac{mitm} attack when a client connects to a site without having a
strict transport security policy. Web browsers (mainly Chrome) mitigate this
vulnerability with an \emph{\ac{hsts} preload list} of domains that are always
treated as if they have such a policy in place. This approach cannot scale as-is
to all \ac{https} sites and thus only protects a limited set of sites.

\ac{https} Everywhere is a Web browser extension that rewrites HTTP requests to
\ac{https} requests for certain sites confirmed to serve over
\ac{https}.\footnote{\url{https://www.eff.org/https-everywhere/}} Because
additions to the list of confirmed \ac{https} sites are typically suggested to
the developers by members of the public, there is a delay between when a site
newly deploys \ac{https} and when \ac{https} Everywhere enforces \ac{https}
connections for that site. Similarly to \ac{hsts} preload lists, this approach
cannot scale to the entire Web and thus protects a limited set of sites (though
larger than does \ac{hsts} preloading).

Smart \ac{https} is a Web browser extension that simply rewrites \emph{all} HTTP
requests to \ac{https} requests, falling back to HTTP if it encounters an error
during the \ac{https} connection
attempt.\footnote{\url{https://mybrowseraddon.com/smart-https.html}} Thus, it
adds significant latency when connecting to HTTP-only sites, as it must first
fall back to HTTP. Smart \ac{https} mitigates this weakness by using the first
response to cache the domain as HTTP or \ac{https}, but this makes \iac{mitm}
attack trivial: an adversary can simply intercept and block the \ac{https}
request to cause Smart \ac{https} to \emph{permanently} log the site as
HTTP-only.

\subsection{Rethinking Certificate Issuance \& Validation}
\label{sec:background:issuance}

Recent work on improving certificate issuance and validation has focused on
providing alternate or additional mechanisms by which domains can authenticate
their public keys to clients. We focus on two approaches: public-key pinning,
and log-based authentication.

Public-key pinning is \iac{tofu} approach that, rather than only enforce the use
of \ac{https}, enforces the use of specific public keys for a site. \ac{hpkp}
allows a domain to specify hashes of public key information for one or more
certificates in the domain's certificate chain~\cite{rfc7469}, while \ac{tack}
allows a domain to specify a public key that the client should treat as a trust
anchor for the domain~\cite{marlinspike2013trust}. For security, both approaches
deliberately make it difficult to prematurely remove or update a pin for a
domain, but this approach can easily make a domain inaccessible due to
misconfiguration or malicious behavior. In 2017, \ac{hpkp} was removed from
Google Chrome due in part to these pitfalls~\cite{palmer2017intent}, and
\ac{tack} was never widely adopted.

Log-based authentication extends the functionality of \ac{ct} to enable logs to
provide or authenticate public-key information for a domain.
AKI~\cite{kim2013accountable}, its successor ARPKI~\cite{basin2014arpki}, and
CIRT~\cite{ryan2014enhanced} have logs directly provide public-key information
for domains and use log proofs to efficiently prove this information.
PoliCert~\cite{szalachowski2014policert} allows domains to specify a rich set of
policies that can emulate systems such as public-key pinning or AKI. While these
systems can provide strong security guarantees (ARPKI, for example, formally
proves protection against \ac{mitm} attacks despite the failure of $n$
\acp{ca}), they require significant changes to the certificate issuance process.
These systems can require multiple \acp{ca} to coordinate to issue a certificate
or easily make a misconfigured domain inaccessible. Moreover, the deployment of
these systems may require a ``flag day'' when issuers switch to the new system
or allow adversaries to downgrade handshakes to the current, vulnerable
\ac{pki}.


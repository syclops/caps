\section{Discussion}
\label{sec:discussion}

\textbf{Data Source Integrity.} \ac{name} relies on the integrity of the data
sources (in our current prototype, Censys, and in the long term, the \ac{ct}
logs). Therefore, \iac{ca} who issues and publicizes an unauthorized certificate
for a domain that does not deploy \ac{https} can cause the domain to be included
in the signal set and thus render the domain inaccessible. This fragility
effectively allows an adversary to ``poison the well'' that \ac{name} relies on,
and execute a denial-of-service attack of indefinite duration with just a single
unauthorized certificate.

Unfortunately, recovering from such an attack requires immediate action. The
affected domain could request that the \ac{ca} revoke the certificate if the
\ac{ca} issued the certificate in error. If the \ac{ca} deliberately misissued
the certificate, the browser vendor could treat the certificate as revoked
(though such action is unlikely unless the domain is popular). The affected
domain could also upgrade to \ac{https}, obtaining enough certificates to
override the misissued certificate. In the event that the domain wishes to
remain without \ac{https}, we can allow domains to set a policy flag that
signals that the domain will communicate over HTTP. The domain, however, must
still send this policy and proof during connection establishment.

\textbf{Size of domain names.}
\begin{compactitem}
\item What can we do about the large number of domain names?
\item Ideas
  \begin{compactitem}
  \item Have separate FSAs for top sites overall or by country (likely small
    changes over time)
  \end{compactitem}
\end{compactitem}

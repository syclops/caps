\section{Deployment Considerations}
\label{sec:discussion}

We now discuss the deployment of \ac{name} in practice. Specifically, we
discuss possible candidates for log aggregators and the expected performance of
\ac{name} in the modern Web.

\paragraph{Candidates for Log Aggregators}

Our design of \ac{name} does not require specific entities to serve as log
aggregators. However, from our analysis in \autoref{sec:analysis:weaknesses},
we conclude that log aggregators should have high availability and widely
known. We believe that browser vendors or public logs would be particularly
suited for these roles. Both already take an active role in ``policing'' the
Web PKI: browser vendors can use updates to avoid known compromised
certificates or updates~\cite{langley2012revocation}, and logs already record
certificates that can uncover \ac{ca} malfeasance~\cite{sleevi2015sustaining}.
Both offer high availability, with logs being held to a minimum of 99\%
availability, minimizing the risk of synchronization issues. Finally, lists
of major browser vendors and logs are already widely known, making it easy to
present clients with a list of available log aggregators to provide policy
proofs.

\paragraph{\ac{name} in the Modern Web}

The use of \ac{https} has increased significantly over the past years
(\autoref{sec:evaluation:https}), in part due to the advent of services such as
Let's Encrypt. Despite this increase, we anticipate the manageable level of
storage and memory overhead of the \ac{name} signaling set in the future for
three reasons. First, \autoref{tab:sample} suggests a sublinear growth in the
size of the signaling set relative to the number of names. Second, despite
\autoref{fig:count:names} showing a large increase in names for which
certificates are being issued, \autoref{fig:updates} suggests that the growth
of names that are accessible on the public Web is much slower. Finally, the
previous points do not take into account the falling cost of disk space and
memory over time, which it even more likely that this overhead will remain
manageable.

An increasing number of Web clients are now mobile devices with limited
computational, memory, and storage resources. While there are performance
tradeoffs that we hope to explore in future work, we can still feasibly deploy
\ac{name} on these devices. As we describe in
\autoref{sec:evaluation:implementation}, smaller forms of the signaling set can
provide limited protection against \ac{mitm} attacks on resource-constrained
devices. Moreover, \autoref{fig:numchain} shows, the added latency of \ac{name}
is small compared to the overall latency, and is mostly due to validating
additional certificate chains or policy proofs.

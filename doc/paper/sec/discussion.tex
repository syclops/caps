\section{Discussion}
\label{sec:discussion}

\steve{\ac{name} relies on the integrity of the data provided by Censys.
  Therefore, an adversary who controls \iac{ca} could issue unauthorized
  certificates for domains that do not deploy \ac{https}, causing them to be
included in the signal set and rendering those sites inaccessible.}

\textbf{Failures of a Probabilistic Approach.} While at first it may seem that
data structures implementing probabilistic membership queries (i.e., Bloom
filters~\cite{bloom1970space} and their variants) could provide much of the
benefits of a signaling set with only a small cost, this approach has several
critical shortcomings. First, Bloom filters allow false positives but no false
negatives. Therefore, while clients are protected from \ac{mitm} attacks that
could result from false negatives, clients may be blocked from accessing sites
that do not deploy \ac{https} because they are false positives in the Bloom
filter. Second, on average, a Bloom filter with false positive probability
\bloomfpp requires approximately $-\ln \bloomfpp/(\ln 2)^2$ bits per inserted
element in the optimal case. Thus for a false positive rate of $p = 0.01$ (which
would still cause 10,000 sites out of 1M sites deploying \ac{https} to be
blocked), we would require 9.59 bits per site on average. Thus for our full data
set, a Bloom filter would require \steve{145 MB}.

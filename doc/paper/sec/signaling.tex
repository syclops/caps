\section{Signaling HTTPS Deployment}
\label{sec:signaling}

In this section, we describe the details of how we signal \ac{https} deployment
in \ac{name}.

\subsection{Data Sources}

In order to signal \ac{https} deployment, we first need to obtain a complete
view of the set of domains that have deployed \ac{https}. We can leverage
\ac{ct} for this task: because Chrome requires that all sites must have
certificates publicly logged in \ac{ct} to be recognized as a valid \ac{https}
site, we can simply download database snapshots of the \ac{ct} public
logs\footnote{\url{https://www.certificate-transparency.org/known-logs}} to
obtain a full set of the sites that deploy \ac{https}.

In order to obtain data about the growth of \ac{https} deployment over time, we
can leverage data sets from Censys~\cite{durumeric2015search}. Among other data
sets, Censys provides scans of the entire IPv4 address space on port 443 (the
default port number for \ac{https}) and the resulting \ac{tls} handshake
attempts. These scans date back to August 2015, allowing us to obtain a picture
of how \ac{https} deployment has increased in the last several years.
Furthermore, Censys's partner projects (used in work predating
Censys~cite{durumeric2013analysis}) provide older scans dating back to June
2012. While these scans do not guarantee a complete view of sites deploying
\ac{https} (missing sites that do not serve \ac{https} over port 443), they
provide insight into how the \ac{https} ecosystem has grown historically.

%To this end, Censys provides scans of the entire IPv4 address space on port 443
%(the default port number for \ac{https}) and the resulting \ac{tls} handshake
%attempts. \steve{TODO: Get \ac{ct} log info and domain names from them.} After
%extracting the data from the \steve{May 18, 2017} scan results, we obtained a
%list of \steve{15.1M} domain names that take up a total of \steve{327.7 MB}.
%This set excludes duplicate domain names as well as common names that are
%invalid DNS names (i.e., those with invalid characters or incorrect wildcard
%placement).

\steve{Maybe include some interesting stats on the domain names we observed}

\subsection{Design}

To construct our \ac{dafsa}, we use an established algorithm for incrementally
constructing and minimizing a \ac{dafsa} from a list of
words~\cite{daciuk2000incremental}. Intuitively, the algorithm works by
constructing branches of a basic prefix tree and then finding and combining
equivalent states. In order to limit the branching that occurs early in the
\ac{dafsa} and to support efficient wildcard name lookups, we construct the
\ac{dafsa} using reverse DNS names (e.g., \texttt{www.example.com} becomes
\texttt{com.example.www}).

Once the \ac{dafsa} has been constructed, we can efficiently represent it using
a variety of techniques. Some techniques, such as storing only the state
transitions and using variable-length encoding as well as flags for common
cases~\cite{daciuk2012smaller}, are established in the finite state methods
literature.

The standard algorithms for constructing \acp{dafsa} use single-character state
transitions. However, we can reduce the size of the underlying directed acyclic
graph by compacting edges as follows: for every arc of the graph (a chain of
state transitions where each intermediate state has one incoming and one
outgoing transition), eliminate the intermediate states and replace the
transitions with a single transition from the start to the end of the arc with a
label of the concatenation of the arc labels. Due to the nature of domain names,
this approach can greatly reduce the size of the \ac{dafsa}, as shown in
\autoref{sec:evaluation}.

We can further reduce the size of the \ac{dafsa} by selectively replacing states
with one incoming transition and $n$ outgoing transitions (alternately, $n$
incoming transitions and one outgoing transition) with $n$ transitions from the
states' predecessors to its successors. In some cases, this approach will
increase the number of unique labels in the \ac{dafsa}, but in all cases
guarantees to eliminate one state and edge from the graph.

Finally, we can more succinctly encode the labels and transitions by using
efficient binary codes. \steve{TODO}

%\subsection{Approaches}

%\steve{This subsection is in note form, and for now is just to summarize the
%approaches I have tried.}

%CRLite makes use of filter cascades (based on Bloom filters) to efficiently
%store the set of all revoked certificates in around 10 MB. However, CRLite's
%approach relies on having access to the set of all known certificates, which
%Censys and the \ac{ct} logs can provide. While it is possible to access many of
%the top-level domain zone files in DNS (including \texttt{.com}), many of the
%registrars of country-specific top-level domains do not publicize their
%information. Moreover, CRLite relies on the \steve{reasonable} assumption that
%only a small minority of certificates will be revoked. By contrast, the rate of
%\ac{https} deployment cannot be bounded by such assumptions, particular with the
%advent of services such as Let's Encrypt, which has already increased the
%\ac{https} deployment rate in its early stages \steve{(todo) wording}.

%Constructing the signal set by simply compressing the set of \ac{https} domains
%is also possible. As with most data compression algorithms, there is a clear
%tradeoff between speed and compression ratio. For example, using lz4
%\steve{(todo) cite} takes less than a second to compress and decompress, but
%only obtains a compression ratio of 2.26 (for a compressed size of 65 MB). Using
%bzip2, on the other hand, takes just over 10 seconds and has a ratio of 3.68 (40
%MB compressed). Using xz took 64 seconds and produced a ratio of 4.45 (33 MB
%compressed), and the best performer, zpaq (with the largest block size), took
%4.5 minutes and produced a ratio of 5.88 (25 MB compressed). Unfortunately,
%decompression with zpaq is slow, taking 7 minutes, and thus cannot be used to
%support real-time signal set checking.

%Succinct data structures provide a way for us to encode the signal set in a
%space-efficient way while supporting efficient membership queries in the
%succinctly encoded state. If we build a trie (also called a prefix tree) based
%on the reversed domain names (capturing the highly repeated use of TLDs), we can
%use the LOUDS (level order unary degree sequence) representation of the prefix
%tree to efficiently encode the tree. In particular, we can represent the full
%signal set in 73.3 MB. \steve{We also tried the use of minimal acyclic finite
  %state automata (MAFSAs), which can represent the same information as a trie in
%fewer states. However, this approach is quite slow for the number of domains we
%want to represent, and there are fewer ways of efficiently encoding MAFSAs,
%which in our case are effectively directed acyclic graphs with labeled edges.}

\subsection{Operational Procedure}

\draft{An external service (such as a browser vendor) periodically retrieves the
latest Censys scan and obtains a list of all domains that have a certificate.
After the \ac{dafsa} is initially constructed, any new domains in subsequent
scans are added to the \ac{dafsa} and the difference is computed. This
difference is then sent to browsers as an update message.}

\draft{Upon receiving an update message, clients verify the update and then
  construct the updated signal set locally. The clients can then begin using the
signal set for deployment checking.}

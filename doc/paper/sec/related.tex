\section{Related Work}
\label{sec:related}

In this section, we discuss related work, covering both the academic literature
and projects from the industry and open-source communities. We first examine
previous work that has addressed the problem of specifying or inferring domain
certificate policy. We then examine work that has addressed the problem of
signaling HTTPS deployment in domains.

\subsection{Specifying and Inferring Domain Policy}

Previous work has proposed the use of domain-specified policies to protect
against \ac{mitm} attacks. In AKI~\cite{kim2013accountable} and its successor
ARPKI~\cite{basin2014arpki}, these policies are embedded within the certificates
themselves and specify attributes such as trusted \acp{ca} and public logs as
well as a ``cool-off period'' that forces potentially malicious certificates to
be visible in logs for some period of time before client can accept them. In
PoliCert~\cite{szalachowski2014policert}, the policies are stored separately in
logs as in \ac{name}. Other work, such as Blockstack~\cite{ali2016blockstack} or
IKP~\cite{matsumoto2017ikp}, has proposed placing policy information into a
blockchain. While IKP's policies operate similarly to those of PoliCert,
Blockstack simply provides the domain's keys directly, and is thus more similar
to DANE~\cite{rfc6698}. Of these proposals, only ARPKI leverages the
relationship between \acp{ca} and logs to ease the burden on the domain during
certificate issuance, but all proposals require the domain to specify its policy
in detail to prevent \ac{mitm} attacks resulting from misbehaving \acp{ca}.

There has also been some work that takes a heuristic approach, using the past
behavior of \acp{ca} in order to determine how clients should trust them. For
example, CAge~\cite{kasten2013cage} propose a system that displays a browser
warning to users for certificates whose names are under a DNS top-level domain
that the issuing \ac{ca} has never signed for before. Perl, Fahl and Smith
propose to outright remove root certificates that have not been used to sign
(either directly or through \ac{ca} delegation) any observed certificates in the
past~\cite{perl2014you}. Both of these proposals use data from Censys or its
predecessors, but do not use any information from the domains themselves.
Because of this, these approaches may encounter situations where they mistake an
attempted \ac{mitm} attack with a legitimate, domain-initiated change and thus
block a user from visiting a benign site or expose a user to the \ac{mitm}
attack. Indeed, a previous study of changes in the Web \ac{pki}'s trust graph
suggests that legitimate changes and attacks share properties to the point that
they cannot be easily distinguished~\cite{amann2013no}.

\subsection{Signaling \ac{https} Deployment}

Though \ac{hsts}~\cite{rfc6797} enforces the use of \ac{https} for a site after
a client has connected with that site over \ac{https} for the first time,
previous work has acknowledged the difficulty (in terms of both storage and
latency overhead) of identifying \ac{https}-deploying domains before the first
visit. The prevailing solution among browser vendors is \emph{\ac{hsts}
preloading}~\cite{keeler2012preloading}, in which browser vendors include a list
of the most popular sites that have deployed \ac{https}. These lists are
incomplete (covering only \steve{TODO: fill in} names in Chrome), domains must
explicitly opt into adding themselves. As indicated by vendor-provided
guidelines,\footnote{\url{https://hstspreload.org/}} this process is prone to
mistakes by domain operators that lead to site inaccessibility.

As described in \steve{TODO: refer to previous section if mentioned elsewhere},
most approaches to disseminating certificate revocation information fail for
signaling because they rely on assumptions that do not hold in the signaling
problem. In particular, the use of \acp{crl}~\cite{rfc5280} does not scale to
the number of \ac{https}-deploying sites because we cannot assume that the
number of sites deploying \ac{https} will remain small compared to the overall
number of sites. More efficient versions of \acp{crl} also do not work: an
approach similar to Chrome's CRLsets~\cite{langley2012revocation} suffers from
the same incompleteness and opt-in hurdles mentioned above, and an approach
similar to CRLite~\cite{larisch2017crlite} relies on knowledge of all domain
names, which is infeasible in today's DNS due to many country code top-level
domains keeping their registries private. Approaches such as using 2-3
trees~\cite{naor1998certificate} or optimized Merkle hash
trees~\cite{laurie2012revocation}, while providing efficient proofs of
\ac{https} deployment, require online checks by clients, greatly increasing the
latency of connection establishment.
